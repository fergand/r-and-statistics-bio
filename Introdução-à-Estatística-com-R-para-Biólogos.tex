% Options for packages loaded elsewhere
\PassOptionsToPackage{unicode}{hyperref}
\PassOptionsToPackage{hyphens}{url}
\PassOptionsToPackage{dvipsnames,svgnames,x11names}{xcolor}
%
\documentclass[
  12pt,
  letterpaper,
  DIV=11,
  numbers=noendperiod]{scrreprt}

\usepackage{amsmath,amssymb}
\usepackage{iftex}
\ifPDFTeX
  \usepackage[T1]{fontenc}
  \usepackage[utf8]{inputenc}
  \usepackage{textcomp} % provide euro and other symbols
\else % if luatex or xetex
  \usepackage{unicode-math}
  \defaultfontfeatures{Scale=MatchLowercase}
  \defaultfontfeatures[\rmfamily]{Ligatures=TeX,Scale=1}
\fi
\usepackage{lmodern}
\ifPDFTeX\else  
    % xetex/luatex font selection
    \setmainfont[]{Times New Roman}
\fi
% Use upquote if available, for straight quotes in verbatim environments
\IfFileExists{upquote.sty}{\usepackage{upquote}}{}
\IfFileExists{microtype.sty}{% use microtype if available
  \usepackage[]{microtype}
  \UseMicrotypeSet[protrusion]{basicmath} % disable protrusion for tt fonts
}{}
\makeatletter
\@ifundefined{KOMAClassName}{% if non-KOMA class
  \IfFileExists{parskip.sty}{%
    \usepackage{parskip}
  }{% else
    \setlength{\parindent}{0pt}
    \setlength{\parskip}{6pt plus 2pt minus 1pt}}
}{% if KOMA class
  \KOMAoptions{parskip=half}}
\makeatother
\usepackage{xcolor}
\setlength{\emergencystretch}{3em} % prevent overfull lines
\setcounter{secnumdepth}{5}
% Make \paragraph and \subparagraph free-standing
\makeatletter
\ifx\paragraph\undefined\else
  \let\oldparagraph\paragraph
  \renewcommand{\paragraph}{
    \@ifstar
      \xxxParagraphStar
      \xxxParagraphNoStar
  }
  \newcommand{\xxxParagraphStar}[1]{\oldparagraph*{#1}\mbox{}}
  \newcommand{\xxxParagraphNoStar}[1]{\oldparagraph{#1}\mbox{}}
\fi
\ifx\subparagraph\undefined\else
  \let\oldsubparagraph\subparagraph
  \renewcommand{\subparagraph}{
    \@ifstar
      \xxxSubParagraphStar
      \xxxSubParagraphNoStar
  }
  \newcommand{\xxxSubParagraphStar}[1]{\oldsubparagraph*{#1}\mbox{}}
  \newcommand{\xxxSubParagraphNoStar}[1]{\oldsubparagraph{#1}\mbox{}}
\fi
\makeatother

\usepackage{color}
\usepackage{fancyvrb}
\newcommand{\VerbBar}{|}
\newcommand{\VERB}{\Verb[commandchars=\\\{\}]}
\DefineVerbatimEnvironment{Highlighting}{Verbatim}{commandchars=\\\{\}}
% Add ',fontsize=\small' for more characters per line
\usepackage{framed}
\definecolor{shadecolor}{RGB}{241,243,245}
\newenvironment{Shaded}{\begin{snugshade}}{\end{snugshade}}
\newcommand{\AlertTok}[1]{\textcolor[rgb]{0.68,0.00,0.00}{#1}}
\newcommand{\AnnotationTok}[1]{\textcolor[rgb]{0.37,0.37,0.37}{#1}}
\newcommand{\AttributeTok}[1]{\textcolor[rgb]{0.40,0.45,0.13}{#1}}
\newcommand{\BaseNTok}[1]{\textcolor[rgb]{0.68,0.00,0.00}{#1}}
\newcommand{\BuiltInTok}[1]{\textcolor[rgb]{0.00,0.23,0.31}{#1}}
\newcommand{\CharTok}[1]{\textcolor[rgb]{0.13,0.47,0.30}{#1}}
\newcommand{\CommentTok}[1]{\textcolor[rgb]{0.37,0.37,0.37}{#1}}
\newcommand{\CommentVarTok}[1]{\textcolor[rgb]{0.37,0.37,0.37}{\textit{#1}}}
\newcommand{\ConstantTok}[1]{\textcolor[rgb]{0.56,0.35,0.01}{#1}}
\newcommand{\ControlFlowTok}[1]{\textcolor[rgb]{0.00,0.23,0.31}{\textbf{#1}}}
\newcommand{\DataTypeTok}[1]{\textcolor[rgb]{0.68,0.00,0.00}{#1}}
\newcommand{\DecValTok}[1]{\textcolor[rgb]{0.68,0.00,0.00}{#1}}
\newcommand{\DocumentationTok}[1]{\textcolor[rgb]{0.37,0.37,0.37}{\textit{#1}}}
\newcommand{\ErrorTok}[1]{\textcolor[rgb]{0.68,0.00,0.00}{#1}}
\newcommand{\ExtensionTok}[1]{\textcolor[rgb]{0.00,0.23,0.31}{#1}}
\newcommand{\FloatTok}[1]{\textcolor[rgb]{0.68,0.00,0.00}{#1}}
\newcommand{\FunctionTok}[1]{\textcolor[rgb]{0.28,0.35,0.67}{#1}}
\newcommand{\ImportTok}[1]{\textcolor[rgb]{0.00,0.46,0.62}{#1}}
\newcommand{\InformationTok}[1]{\textcolor[rgb]{0.37,0.37,0.37}{#1}}
\newcommand{\KeywordTok}[1]{\textcolor[rgb]{0.00,0.23,0.31}{\textbf{#1}}}
\newcommand{\NormalTok}[1]{\textcolor[rgb]{0.00,0.23,0.31}{#1}}
\newcommand{\OperatorTok}[1]{\textcolor[rgb]{0.37,0.37,0.37}{#1}}
\newcommand{\OtherTok}[1]{\textcolor[rgb]{0.00,0.23,0.31}{#1}}
\newcommand{\PreprocessorTok}[1]{\textcolor[rgb]{0.68,0.00,0.00}{#1}}
\newcommand{\RegionMarkerTok}[1]{\textcolor[rgb]{0.00,0.23,0.31}{#1}}
\newcommand{\SpecialCharTok}[1]{\textcolor[rgb]{0.37,0.37,0.37}{#1}}
\newcommand{\SpecialStringTok}[1]{\textcolor[rgb]{0.13,0.47,0.30}{#1}}
\newcommand{\StringTok}[1]{\textcolor[rgb]{0.13,0.47,0.30}{#1}}
\newcommand{\VariableTok}[1]{\textcolor[rgb]{0.07,0.07,0.07}{#1}}
\newcommand{\VerbatimStringTok}[1]{\textcolor[rgb]{0.13,0.47,0.30}{#1}}
\newcommand{\WarningTok}[1]{\textcolor[rgb]{0.37,0.37,0.37}{\textit{#1}}}

\providecommand{\tightlist}{%
  \setlength{\itemsep}{0pt}\setlength{\parskip}{0pt}}\usepackage{longtable,booktabs,array}
\usepackage{calc} % for calculating minipage widths
% Correct order of tables after \paragraph or \subparagraph
\usepackage{etoolbox}
\makeatletter
\patchcmd\longtable{\par}{\if@noskipsec\mbox{}\fi\par}{}{}
\makeatother
% Allow footnotes in longtable head/foot
\IfFileExists{footnotehyper.sty}{\usepackage{footnotehyper}}{\usepackage{footnote}}
\makesavenoteenv{longtable}
\usepackage{graphicx}
\makeatletter
\newsavebox\pandoc@box
\newcommand*\pandocbounded[1]{% scales image to fit in text height/width
  \sbox\pandoc@box{#1}%
  \Gscale@div\@tempa{\textheight}{\dimexpr\ht\pandoc@box+\dp\pandoc@box\relax}%
  \Gscale@div\@tempb{\linewidth}{\wd\pandoc@box}%
  \ifdim\@tempb\p@<\@tempa\p@\let\@tempa\@tempb\fi% select the smaller of both
  \ifdim\@tempa\p@<\p@\scalebox{\@tempa}{\usebox\pandoc@box}%
  \else\usebox{\pandoc@box}%
  \fi%
}
% Set default figure placement to htbp
\def\fps@figure{htbp}
\makeatother
% definitions for citeproc citations
\NewDocumentCommand\citeproctext{}{}
\NewDocumentCommand\citeproc{mm}{%
  \begingroup\def\citeproctext{#2}\cite{#1}\endgroup}
\makeatletter
 % allow citations to break across lines
 \let\@cite@ofmt\@firstofone
 % avoid brackets around text for \cite:
 \def\@biblabel#1{}
 \def\@cite#1#2{{#1\if@tempswa , #2\fi}}
\makeatother
\newlength{\cslhangindent}
\setlength{\cslhangindent}{1.5em}
\newlength{\csllabelwidth}
\setlength{\csllabelwidth}{3em}
\newenvironment{CSLReferences}[2] % #1 hanging-indent, #2 entry-spacing
 {\begin{list}{}{%
  \setlength{\itemindent}{0pt}
  \setlength{\leftmargin}{0pt}
  \setlength{\parsep}{0pt}
  % turn on hanging indent if param 1 is 1
  \ifodd #1
   \setlength{\leftmargin}{\cslhangindent}
   \setlength{\itemindent}{-1\cslhangindent}
  \fi
  % set entry spacing
  \setlength{\itemsep}{#2\baselineskip}}}
 {\end{list}}
\usepackage{calc}
\newcommand{\CSLBlock}[1]{\hfill\break\parbox[t]{\linewidth}{\strut\ignorespaces#1\strut}}
\newcommand{\CSLLeftMargin}[1]{\parbox[t]{\csllabelwidth}{\strut#1\strut}}
\newcommand{\CSLRightInline}[1]{\parbox[t]{\linewidth - \csllabelwidth}{\strut#1\strut}}
\newcommand{\CSLIndent}[1]{\hspace{\cslhangindent}#1}

% header.tex

% -------------------------------------------------
% 1. Layout geral e margens
% -------------------------------------------------
\usepackage{geometry}
\geometry{margin=1in}
\usepackage{float}

% -------------------------------------------------
% 2. Matemática e teoremas
% -------------------------------------------------
\usepackage{amsmath, amsthm, amssymb}

% -------------------------------------------------
% 3. Fonte, espaçamento e microtipografia
% -------------------------------------------------
\usepackage{lmodern}
\usepackage{textcomp}
\usepackage{microtype}
\usepackage{setspace}
\onehalfspacing

% -------------------------------------------------
% 4. Primeiro parágrafo e identação
% -------------------------------------------------
\usepackage{indentfirst}
\setlength{\parindent}{1.25cm}
\setlength{\parskip}{0pt}

% -------------------------------------------------
% 5. Sumário (toc) customizado
% -------------------------------------------------
\usepackage{tocloft}
\renewcommand{\cftpartleader}{\cftdotfill{\cftdotsep}}
\renewcommand{\cftchapleader}{\cftdotfill{\cftdotsep}}

% -------------------------------------------------
% 6. Tabelas
% -------------------------------------------------
\usepackage{booktabs}
\usepackage{longtable}
\usepackage{array}
\usepackage{multirow}
\usepackage{tabu}

% -------------------------------------------------
% 7. Backref (referências com página)
% -------------------------------------------------
\usepackage[brazilian,hyperpageref]{backref}

% -------------------------------------------------
% 8. Verbose & blocos de código
% -------------------------------------------------
\usepackage{fvextra}
\DefineVerbatimEnvironment{verbatim}{Verbatim}{
	breaklines=true,
	breakanywhere=true,
	breaksymbolleft={}, 
	breaksymbolright={},
	fontsize=\small
}
% se você quiser que o ambiente Shaded (usado pelo quarto)
% também ganhe fonte menor:
\AtBeginEnvironment{Shaded}{\small}

% e se algum chunk acabar usando listings:
\usepackage{listings}
\lstset{
	breaklines=true,
	breakatwhitespace=true,
	postbreak=\mbox{\textcolor{gray}{$\hookrightarrow$}\space},
	basicstyle=\ttfamily\small,
	columns=fullflexible
}
\usepackage{booktabs}
\usepackage{longtable}
\usepackage{array}
\usepackage{multirow}
\usepackage{wrapfig}
\usepackage{float}
\usepackage{colortbl}
\usepackage{pdflscape}
\usepackage{tabu}
\usepackage{threeparttable}
\usepackage{threeparttablex}
\usepackage[normalem]{ulem}
\usepackage{makecell}
\usepackage{xcolor}
\KOMAoption{captions}{tableheading}
\makeatletter
\@ifpackageloaded{bookmark}{}{\usepackage{bookmark}}
\makeatother
\makeatletter
\@ifpackageloaded{caption}{}{\usepackage{caption}}
\AtBeginDocument{%
\ifdefined\contentsname
  \renewcommand*\contentsname{Índice}
\else
  \newcommand\contentsname{Índice}
\fi
\ifdefined\listfigurename
  \renewcommand*\listfigurename{Lista de Figuras}
\else
  \newcommand\listfigurename{Lista de Figuras}
\fi
\ifdefined\listtablename
  \renewcommand*\listtablename{Lista de Tabelas}
\else
  \newcommand\listtablename{Lista de Tabelas}
\fi
\ifdefined\figurename
  \renewcommand*\figurename{Figura}
\else
  \newcommand\figurename{Figura}
\fi
\ifdefined\tablename
  \renewcommand*\tablename{Tabela}
\else
  \newcommand\tablename{Tabela}
\fi
}
\@ifpackageloaded{float}{}{\usepackage{float}}
\floatstyle{ruled}
\@ifundefined{c@chapter}{\newfloat{codelisting}{h}{lop}}{\newfloat{codelisting}{h}{lop}[chapter]}
\floatname{codelisting}{Listagem}
\newcommand*\listoflistings{\listof{codelisting}{Lista de Listagens}}
\makeatother
\makeatletter
\makeatother
\makeatletter
\@ifpackageloaded{caption}{}{\usepackage{caption}}
\@ifpackageloaded{subcaption}{}{\usepackage{subcaption}}
\makeatother

\ifLuaTeX
\usepackage[bidi=basic]{babel}
\else
\usepackage[bidi=default]{babel}
\fi
\babelprovide[main,import]{brazilian}
\ifPDFTeX
\else
\babelfont{rm}[]{Times New Roman}
\fi
% get rid of language-specific shorthands (see #6817):
\let\LanguageShortHands\languageshorthands
\def\languageshorthands#1{}
\usepackage{bookmark}

\IfFileExists{xurl.sty}{\usepackage{xurl}}{} % add URL line breaks if available
\urlstyle{same} % disable monospaced font for URLs
\hypersetup{
  pdftitle={Introdução à Estatística com R para Biólogos},
  pdfauthor={Fernando Andrade},
  pdflang={pt-br},
  colorlinks=true,
  linkcolor={blue},
  filecolor={Maroon},
  citecolor={Blue},
  urlcolor={Blue},
  pdfcreator={LaTeX via pandoc}}


\title{Introdução à Estatística com R para Biólogos}
\author{Fernando Andrade}
\date{}

\begin{document}
\maketitle

\renewcommand*\contentsname{Sumário}
{
\hypersetup{linkcolor=}
\setcounter{tocdepth}{2}
\tableofcontents
}

\bookmarksetup{startatroot}

\chapter*{Prefácio}\label{prefuxe1cio}
\addcontentsline{toc}{chapter}{Prefácio}

\markboth{Prefácio}{Prefácio}

Essa apostila foi criada para ser um guia abrangente e conceitual para
pesquisadores de mestrado e doutorado na área de Ciências Biológicas e
Zootecnia, com um foco particular em estudos de aves. Reconhecendo que
muitos biólogos possuem uma base sólida em suas áreas de especialidade
mas podem não ter uma formação apropriada em programação e estatística
teórica, este material busca preencher essa lacuna de maneira saudável.
O objetivo não é apresentar um oceano de fórmulas matemáticas, mas sim
construir uma compreensão intuitiva e rigorosa dos princípios
estatísticos e de sua aplicação prática utilizando a linguagem de
programação R, um forte e indispensável aliado nos tempos atuais para
pesquisadores e cientistas de dados.

A estrutura da apostila foi projetada para seguir uma progressão lógica,
começando com as habilidades fundamentais de programação e manipulação
de dados em R, e avançando gradualmente para modelos estatísticos mais
complexos que são frequentemente necessários para pesquisas biológicas,
como os modelos lineares mistos e generalizados mistos. Cada capítulo
combina explicações conceituais, exemplos práticos contextualizados na
ornitologia e zootecnia, e blocos de códigos em R detalhadamente
explicados.

O pressuposto é que a estatística não seja um obstáculo a ser superado,
mas uma ferramenta poderosa para analisar criteriosamente os dados,
planejar experimentos robustos e extrair conclusões válidas e
significativas do trabalho de pesquisa. Ao final desta jornada, é
esperado que o leitor estará equipado não apenas para analisar seus
próprios dados com confiança, mas também para interpretar criticamente a
literatura científica de sua área.

Zuur et al. (\citeproc{ref-zuurAnalysingEcologicalData2007}{2007}) e
(\citeproc{ref-zuurAnalysingEcologicalData2007}{ZUUR ET AL., 2007})

\part{Ferramentas Essenciais: Dominando o R}

Nesta primeira parte, o foco residirá no entendimento das ferramentas
computacionais essenciais que fundamentam qualquer análise de dados
moderna. Antes de mergulhar nos conceitos estatísticos propriamente
ditos, é necessário desenvolver uma fluência básica no ambiente de
programação que será utilizado.

A abordagem que será adotada aqui prioriza a eficiência e a intuição,
introduzindo o início do ecossistema \texttt{tidyverse}, um conjunto de
pacotes R projetados para trabalhar em harmonia, tornando a manipulação,
exploração e visualização de dados um processo mais humano, lógico e
simples.

Ao dominar essa ferramenta, o leitor transformará tarefas árduas de
preparação de dados em uma parte integrada e poderosa do processo de
descoberta científica.

\chapter{Introdução}\label{introduuxe7uxe3o}

Para começar essa jornada, o primeiro passo é configurar o ambiente de
trabalho. Isso envolve a instalação de dois \emph{softwares} distintos,
mas que trabalham juntos: R e RStudio. Compreender o funcionamento de
cada um e como eles se interagem é fundamental para as próximas etapas.

\section{O que são R e RStudio?}\label{o-que-suxe3o-r-e-rstudio}

É comum que iniciantes confundam R e RStudio, mas esta distinção é
crucial para o processo.

\begin{itemize}
\tightlist
\item
  \textbf{R} é a linguagem de programação e o ambiente de software para
  computação estatística e gráficos. Pode-se pensar que é o ``motor''
  que executa todos os cálculos, análises e gera os gráficos. Além de
  tudo, é um projeto de código aberto, gratuito e mantido por uma vasta
  comunidade de desenvolvedores e estatísticos ao redor do mundo.
\item
  \textbf{RStudio} é um Ambiente de Desenvolvimento Integrado (IDE, do
  inglês \emph{Integrated Development Environment}). Se o R é o motor do
  carro, o RStudio é o painel, o volante, e todo o interior que torna a
  condução do carro uma experiência agradável e gerenciável. O RStudio
  fornece uma interface gráfica e amigável que organiza o trabalho em R,
  facilitando a escrita de \emph{scripts} (arquivos de códigos), a
  visualização de gráficos, o gerenciamento de pacotes (bibliotecas) e
  muito mais. Embora seja possível utilizar o R sem o RStudio, a
  utilização do RStudio é fortemente recomendada, pois deixa o processo
  de análise muito mais interativo e organizado.
\end{itemize}

\section{Instalação passo a passo}\label{instalauxe7uxe3o-passo-a-passo}

A instalação adequada dos programas é um pré-requisito crucial. A ordem
de instalação é importante: \textbf{R deve ser instalado antes do
RStudio}.

\begin{enumerate}
\def\labelenumi{\arabic{enumi}.}
\tightlist
\item
  Instalando o R:
\end{enumerate}

\begin{itemize}
\tightlist
\item
  Acesse o \href{https://cran.r-project.org/}{site} do
  \emph{Comprehensive R Archive Network} (CRAN), que é o repositório
  oficial para o R e seus pacotes.
\item
  Na página inicial, selecione o link de download para o seu sistema
  operacional (Linux, macOS ou Windows).
\item
  Siga as instruções para baixar a versão mais recente (``base''). É
  crucial baixar a versão diretamente do CRAN, pois os gerenciadores de
  pacotes de alguns sistemas operacionais (como o \texttt{get-apt} do
  Ubuntu) podem fornecer versões desatualizadas.
\item
  Execute o arquivo de instalação baixado e siga as instruções padrão,
  aceitando as configurações padrão.
\end{itemize}

\begin{enumerate}
\def\labelenumi{\arabic{enumi}.}
\setcounter{enumi}{1}
\tightlist
\item
  Instalando o RStudio:
\end{enumerate}

\begin{itemize}
\tightlist
\item
  Após a instalação do R, acesse o
  \href{https://posit.co/download/rstudio-desktop/}{site da Posit} e
  clique para baixar a versão gratuita do RStudio Desktop.
\item
  Baixe o instalador apropriado para o seu sistema operacional.
\item
  Execute o arquivo de instalação. O RStudio detectará automaticamente a
  instalação do R existente.
\end{itemize}

\section{Navegando na interface do
RStudio}\label{navegando-na-interface-do-rstudio}

Ao abrir o RStudio pela primeira vez, a interface se apresenta dividida
em quatro painéis ou quadrantes principais, cada um com uma função
específica:

\begin{enumerate}
\def\labelenumi{\arabic{enumi}.}
\tightlist
\item
  \textbf{Editor de \emph{scripts}} (Superior esquerdo): Este é o seu
  principal espaço de trabalho. Aqui, você escreverá e salvará seus
  \emph{scripts} R (arquivos com extensão \texttt{.R}). Trabalhar em um
  \emph{script}, em vez de digitar comandos diretamente no console, é a
  base da ciência reprodutível, pois permite salvar, comentar e
  reutilizar seu código.
\item
  \textbf{Console} (Inferior esquerdo): O console é o código R é
  efetivamente executado. Você pode digitar os comandos diretamente nele
  para testes rápidos ou executar linhas de códigos do seu \emph{script}
  (utilizando o atalho \texttt{Ctrl+Enter}). A saída dos comandos também
  aparecerá aqui.
\item
  \textbf{Ambiente e Histórico} (Superior direito): A aba
  \emph{Environment} mostra todos os objetos (como \emph{datasets},
  variáveis, etc.) que foram criadas na sessão atual do R. Já a aba
  \emph{History} mantém um registro de todos os comandos utilizados.
\item
  \textbf{Arquivos, Gráficos, Pacotes e Ajuda} (Inferior direita): Este
  painel multifuncional permite navegar pelos arquivos do seu computador
  (\emph{Files}) , visualizar gráficos gerados (\emph{Plots}), gerenciar
  pacotes instalados (\emph{Packages}), e acessar documentações de ajuda
  do R (\emph{Help}).
\end{enumerate}

É importante salientar que o RStudio permite customizações, como a
alteração das posições dos painéis.

\section{O conceito de pacotes}\label{o-conceito-de-pacotes}

A grande força do R reside em seu ecossistema de pacotes. Um pacote é a
coleção de funções, dados e documentação que estende as capacidades
iniciais do R. Para qualquer tarefa estatística ou de manipulação de
dados que se possa imaginar, provavelmente existe algum pacote que a
facilita.

\subsection{Instalando e Carregando Pacotes
Essenciais}\label{instalando-e-carregando-pacotes-essenciais}

Existe uma distinção básica a ser realizada entre instalar e carregar um
pacote.

\begin{itemize}
\tightlist
\item
  \textbf{Instalação}: É o ato de baixar o pacote do CRAN e instalá-lo
  no computador. Isso é realizado apenas uma vez para cada pacote.
\item
  \textbf{Carregamento}: É o ato de carregar o pacote instalado em sua
  sessão do R de forma que as funções adicionais fiquem disponíveis para
  uso. Isso precisa ser feito toda vez que uma sessão no R é iniciada.
\end{itemize}

Para este material, os pacotes centrais são: \texttt{tidyverse},
\texttt{lme4}, \texttt{lmerTest} e \texttt{nlme}. Um dos métodos para
instalar pacotes R no computador é por meio da função
\texttt{install.packages()}:

\begin{Shaded}
\begin{Highlighting}[]
\CommentTok{\# Instala o pacote tidyverse, que inclui dplyr, ggplot2 e outros}
\FunctionTok{install.packages}\NormalTok{(}\StringTok{"tidyverse"}\NormalTok{)}

\CommentTok{\# Instala o pacote para modelos lineares mistos}
\FunctionTok{install.packages}\NormalTok{(}\StringTok{"lme4"}\NormalTok{)}

\CommentTok{\# Instala outros pacotes para modelos mistos}
\FunctionTok{install.packages}\NormalTok{(}\StringTok{"lmerTest"}\NormalTok{)}
\FunctionTok{install.packages}\NormalTok{(}\StringTok{"nlme"}\NormalTok{)}
\end{Highlighting}
\end{Shaded}

Após a instalação, para usar as funções de um pacote, é preciso
carregá-lo com a função \texttt{library()}:

\begin{Shaded}
\begin{Highlighting}[]
\FunctionTok{library}\NormalTok{(tidyverse)}
\end{Highlighting}
\end{Shaded}

Cabe ressaltar que, ao longo do uso de diversos pacotes, podem ocorrer
conflitos de funções com o mesmo nome. Nesses casos, a solução mais
prática é utilizar a notação \texttt{pacote::funcao} para indicar
explicitamente ao R de qual biblioteca desejamos chamar a função.

\section{Diretório de Trabalho e Projetos
RStudio}\label{diretuxf3rio-de-trabalho-e-projetos-rstudio}

O diretório de trabalho é a pasta no seu computador onde o R irá
procurar por arquivos para ler e onde, também, salvará os arquivos
criados (como gráficos, \emph{scripts} e \emph{datasets} modificados). É
possível identificar o diretório atual através do comando
\texttt{getwd()} e, embora também seja possível defini-la manualmente
com a função \texttt{setwd("caminho/para/sua/pasta")}, essa prática não
é aconselhável, visto que o uso de caminhos de arquivos absolutos torna
o código não portável; ou seja, ele não irá funcionar se você mover a
pasta do projeto ou tentá-la executá-lo em outro computador.

A solução moderna e robusta para esse problema é a utilização de
\textbf{Projetos RStudio}. Um projeto RStudio (extensão \texttt{.Rproj})
é um arquivo que você cria dentro de uma pasta do seu projeto de
pesquisa. Ao abrir um projeto, o RStudio automaticamente define o
diretório de trabalho para aquela pasta. Isso garante que todos os
caminhos de arquivo do seu código possam ser relativos à raiz do
projeto, tornando sua análise totalmente reprodutível e compartilhável
de forma eficaz. Outra maneira de criar projetos é através do próprio
RStudio, através das seguintes instruções
\texttt{File\ \textgreater{}\ New\ Project\ \textgreater{}\ New\ Directory\ \textgreater{}\ New\ Project}
e nesta última etapa, você escolherá um nome para o projeto e a pasta de
sua pesquisa, finalizando em \texttt{Create\ Project}. A criação de um
projeto para cada análise de pesquisa é uma prática fundamental para a
organização e a reprodutibilidade científica.

\section{R Básico}\label{r-buxe1sico}

A leitura desta sessão é aconselhada para o leitor que nunca teve
contato com o R. Os tópicos introduzidos são especiais para a
compreensão do que é um \emph{dataframe}, a estrutura dos
\emph{datasets} dentro do R, e quais operações estarão sendo realizadas
quando estivermos efetuando filtragens e modificações de suas colunas.
Também são importantes para a compreensão do que é uma função no R.

\subsection{Operadores Matemáticos}\label{operadores-matemuxe1ticos}

Os operadores matemáticos, também conhecidos por operadores binários,
dentro do ambiente R soam como familiares. A Tabela \ref{tbl-op-mat}
exibe os operadores mais básicos utilizados.

Para exemplificar como efetuar cálculos de expressões matemáticas no R,
suponha que desenhamos calcular o valor de:
\[2\times 2 + \frac{4 + 4}{2}.\] Para isso, escrevemos
\texttt{2*2\ +\ (4+4)/2} no console para determinarmos o resultado

\begin{Shaded}
\begin{Highlighting}[]
\DecValTok{2}\SpecialCharTok{*}\DecValTok{2} \SpecialCharTok{+}\NormalTok{ (}\DecValTok{4}\SpecialCharTok{+}\DecValTok{4}\NormalTok{)}\SpecialCharTok{/}\DecValTok{2}
\end{Highlighting}
\end{Shaded}

\begin{verbatim}
[1] 8
\end{verbatim}

\begin{longtable}[t]{ll}

\caption{\label{tbl-op-mat}Operadores matemáticos básicos.}

\tabularnewline

\toprule
Operadores & Descrição\\
\midrule
\cellcolor{gray!10}{+} & \cellcolor{gray!10}{Adição}\\
- & Subtração\\
\cellcolor{gray!10}{*} & \cellcolor{gray!10}{Multiplicação}\\
/ & Divisão\\
\cellcolor{gray!10}{\textasciicircum{}} & \cellcolor{gray!10}{Exponenciação}\\
\bottomrule

\end{longtable}

\subsection{Objetos e funções}\label{objetos-e-funuxe7uxf5es}

O R permite guardar valores dentro de um \textbf{objeto}. Um objeto é
simplesmente um nome que guarda uma determinada informação na memória do
computador, que é criado por meio do operador \texttt{\textless{}-}.
Veja que no código a seguir

\begin{Shaded}
\begin{Highlighting}[]
\NormalTok{x }\OtherTok{\textless{}{-}} \DecValTok{10} \CommentTok{\# Salvando "10" em "x"}
\NormalTok{x       }\CommentTok{\# Avaliando o objeto "x"}
\end{Highlighting}
\end{Shaded}

\begin{verbatim}
[1] 10
\end{verbatim}

\noindent foi salvo que a informação que \texttt{x} carrega é o valor
\texttt{10}. Portanto, toda vez que o objeto \texttt{x} for avaliado, o
R irá devolver o valor \texttt{10}.

É importante ressaltar que há regras para a nomeação dos objetos, dentre
elas, não começar com números. Assim, todos os seguintes exemplos são
permitidos: \texttt{x\ \textless{}-\ 1}, \texttt{x1\ \textless{}-\ 1},
\texttt{meu\_objeto\ \textless{}-\ 1},
\texttt{meu.objeto\ \textless{}-\ 1}. Ainda, o R diferencia letras
minúsculas de maiúsculas, então objetos como \texttt{y} e \texttt{Y} são
diferentes.

Enquanto que os objetos são nomes que salvam informações de valores,
\textbf{funções} são nomes que guardam informações de um código R,
retornando algum resultado programado. A sintaxe básica de uma função é
\texttt{nome\_funcao(arg1,\ arg2,\ ...)}. Os valores dentro dos
parênteses são chamados por \textbf{argumentos}, que são informações
necessárias para o bom funcionamento de uma função. Às vezes, uma função
não necessita do fornecimento de argumentos específicos.

Uma função simples, porém útil, é a \texttt{sum()}. Ela consiste em
somar os valores passados em seu argumento. Suponha que desejamos somar
\texttt{1+2+3+4+5}. Assim,

\begin{Shaded}
\begin{Highlighting}[]
\FunctionTok{sum}\NormalTok{(}\DecValTok{1}\NormalTok{,}\DecValTok{2}\NormalTok{,}\DecValTok{3}\NormalTok{,}\DecValTok{4}\NormalTok{,}\DecValTok{5}\NormalTok{)}
\end{Highlighting}
\end{Shaded}

\begin{verbatim}
[1] 15
\end{verbatim}

\noindent é possível reparar que o resultado é 15.

A classe de um objeto é muito importante na programação em R. É a partir
disso que as funções e operadores conseguem entender o que fazer com
cada objeto. Há uma infinidade de classes, dentre as mais conhecidas
são: \texttt{numeric}, \texttt{character}, \texttt{data.frame},
\texttt{logical} e \texttt{factor}. Para averiguar o tipo de classe, a
função \texttt{class()} retorna exatamente a classe do objeto.

\begin{Shaded}
\begin{Highlighting}[]
\FunctionTok{class}\NormalTok{(}\StringTok{"a"}\NormalTok{)}
\end{Highlighting}
\end{Shaded}

\begin{verbatim}
[1] "character"
\end{verbatim}

\begin{Shaded}
\begin{Highlighting}[]
\FunctionTok{class}\NormalTok{(}\DecValTok{1}\NormalTok{)}
\end{Highlighting}
\end{Shaded}

\begin{verbatim}
[1] "numeric"
\end{verbatim}

\begin{Shaded}
\begin{Highlighting}[]
\FunctionTok{class}\NormalTok{(mtcars)}
\end{Highlighting}
\end{Shaded}

\begin{verbatim}
[1] "data.frame"
\end{verbatim}

\begin{Shaded}
\begin{Highlighting}[]
\FunctionTok{class}\NormalTok{(}\ConstantTok{TRUE}\NormalTok{)}
\end{Highlighting}
\end{Shaded}

\begin{verbatim}
[1] "logical"
\end{verbatim}

\subsection{Importanto dados}\label{importanto-dados}

Uma atividade importante para qualquer análise estatística que vier ser
feita no R é importante importar os dados para o ambiente de trabalho,
que ficarão guardados dentro de um objeto no projeto RStudio -- afinal,
como faríamos as análises sem os dados? No contexto da Biologia, isso
costuma significar ler arquivos com medidas de peso, contagens de
indivíduos, medidas de comprimento etc., geralmente armazenados em
formatos de texto (\texttt{.csv} ou \texttt{.tsv}) ou planilhas
(\texttt{.xlsx}). As principais funções para cada ocasião de arquivo
são:

\begin{itemize}
\tightlist
\item
  CSV com cabeçalho:
\end{itemize}

\begin{Shaded}
\begin{Highlighting}[]
\NormalTok{dados }\OtherTok{\textless{}{-}} \FunctionTok{read.csv}\NormalTok{(}\StringTok{"dados.csv"}\NormalTok{,}
  \AttributeTok{header =} \ConstantTok{TRUE}\NormalTok{, }\CommentTok{\# indica que há cabeçalho}
  \AttributeTok{sep    =} \StringTok{","}\NormalTok{,  }\CommentTok{\# separador vírgula}
  \AttributeTok{stringsAsFactors =} \ConstantTok{FALSE} \CommentTok{\# evita conversão automática em fatores}
\NormalTok{)}
\end{Highlighting}
\end{Shaded}

\begin{itemize}
\tightlist
\item
  TXT ou TSV com tabulação:
\end{itemize}

\begin{Shaded}
\begin{Highlighting}[]
\NormalTok{dados }\OtherTok{\textless{}{-}} \FunctionTok{read.delim}\NormalTok{(}\StringTok{"dadostsv"}\NormalTok{, }
  \AttributeTok{header =} \ConstantTok{TRUE}\NormalTok{, }
  \AttributeTok{sep    =} \StringTok{"}\SpecialCharTok{\textbackslash{}t}\StringTok{"}
\NormalTok{)}
\end{Highlighting}
\end{Shaded}

\begin{itemize}
\tightlist
\item
  Planilhas no Excel (arquivos \texttt{.xlsx}):
\end{itemize}

\begin{Shaded}
\begin{Highlighting}[]
\NormalTok{dados }\OtherTok{\textless{}{-}}\NormalTok{ readxl}\SpecialCharTok{::}\FunctionTok{read\_excel}\NormalTok{(}\StringTok{"dados.xlsx"}\NormalTok{,}
  \AttributeTok{sheet =} \StringTok{"Planilha1"} \CommentTok{\# aqui você escolhe a planilha a ser lida}
\NormalTok{)}
\end{Highlighting}
\end{Shaded}

\noindent Ressaltamos, neste caso, a necessidade da utilização da
biblioteca \texttt{readxl} para que seja possível lermos planilhas no R.

\subsection{\texorpdfstring{Vetores e \emph{Data
frames}}{Vetores e Data frames}}\label{vetores-e-data-frames}

Vetores são uma estrutura fundamental dentro do R, em especial, é a
partir deles que os \emph{data frames} são construídos. Por definição,
são conjuntos indexados de valores e para criá-los, basta utilizar a
função \texttt{c()} com valores separados por vírgula (ex.:
\texttt{c(1,2,4,10)}). Para acessar um valor dentro de um determinado
vetor, utiliza-se os colchetes \texttt{{[}{]}}:

\begin{Shaded}
\begin{Highlighting}[]
\NormalTok{vetor }\OtherTok{\textless{}{-}} \FunctionTok{c}\NormalTok{(}\StringTok{"a"}\NormalTok{, }\StringTok{"b"}\NormalTok{, }\StringTok{"c"}\NormalTok{)}

\CommentTok{\# Acessando valor "b"}
\NormalTok{vetor[}\DecValTok{2}\NormalTok{]}
\end{Highlighting}
\end{Shaded}

\begin{verbatim}
[1] "b"
\end{verbatim}

\noindent Um vetor só pode guardar um tipo de objeto e ele terá sempre a
mesma classe dos objetos que guarda. Caso tentarmos misturar duas
classes, o R vai apresentar o comportamento conhecido como
\textbf{coerção}.

\begin{Shaded}
\begin{Highlighting}[]
\FunctionTok{class}\NormalTok{(}\FunctionTok{c}\NormalTok{(}\DecValTok{1}\NormalTok{,}\DecValTok{2}\NormalTok{,}\DecValTok{3}\NormalTok{))}
\end{Highlighting}
\end{Shaded}

\begin{verbatim}
[1] "numeric"
\end{verbatim}

\begin{Shaded}
\begin{Highlighting}[]
\FunctionTok{class}\NormalTok{(vetor)}
\end{Highlighting}
\end{Shaded}

\begin{verbatim}
[1] "character"
\end{verbatim}

\begin{Shaded}
\begin{Highlighting}[]
\FunctionTok{class}\NormalTok{(}\FunctionTok{c}\NormalTok{(}\DecValTok{1}\NormalTok{,}\DecValTok{2}\NormalTok{,}\StringTok{"a"}\NormalTok{,}\StringTok{"b"}\NormalTok{))}
\end{Highlighting}
\end{Shaded}

\begin{verbatim}
[1] "character"
\end{verbatim}

\noindent Neste caso, todos os elementos do vetor se transformaram em
texto.

Assim, também, \emph{data frames} são de extrema importância no R, visto
que são os objetos que guardam os dados e são equivalentes a uma
planilha do Excel. A principal característica é possuir linha e colunas.
Em geral, as colunas são vetores de mesmo tamanho (ou dimensão). Um
valor específico de um \emph{data frame} pode ser acessado, também, via
colchetes \texttt{{[}{]}}:

\begin{Shaded}
\begin{Highlighting}[]
\FunctionTok{class}\NormalTok{(mtcars)}
\end{Highlighting}
\end{Shaded}

\begin{verbatim}
[1] "data.frame"
\end{verbatim}

\begin{Shaded}
\begin{Highlighting}[]
\NormalTok{mtcars[}\DecValTok{1}\NormalTok{,}\DecValTok{2}\NormalTok{]}
\end{Highlighting}
\end{Shaded}

\begin{verbatim}
[1] 6
\end{verbatim}

\noindent \texttt{mtcars} é um conjunto de dados muito conhecido na
comunidade R.

\subsection{Fatores}\label{fatores}

Fatores são uma classe de objetos no R criada para representar variáveis
categóricas numericamente. A característica que define essa classe é o
atributo \texttt{levels}, que representam as possíveis categorias de uma
variável categórica.

A título de exemplificação, considere o objeto \texttt{sexo} que contém
as informações do sexo de uma pessoa. As possibilidades são: \texttt{F}
(feminino) e \texttt{M} (masculino). Por padrão, o R interpreta essa
variável como texto (\emph{character}), no entanto, é possível
transformá-la em fator por meio da função \texttt{as.factor()}.

\begin{Shaded}
\begin{Highlighting}[]
\NormalTok{sexo }\OtherTok{\textless{}{-}} \FunctionTok{c}\NormalTok{(}\StringTok{"F"}\NormalTok{, }\StringTok{"F"}\NormalTok{, }\StringTok{"M"}\NormalTok{, }\StringTok{"M"}\NormalTok{, }\StringTok{"F"}\NormalTok{)}
\FunctionTok{class}\NormalTok{(sexo)}
\end{Highlighting}
\end{Shaded}

\begin{verbatim}
[1] "character"
\end{verbatim}

\begin{Shaded}
\begin{Highlighting}[]
\CommentTok{\# Transformando em fator}
\FunctionTok{class}\NormalTok{(}\FunctionTok{as.factor}\NormalTok{(sexo))}
\end{Highlighting}
\end{Shaded}

\begin{verbatim}
[1] "factor"
\end{verbatim}

\begin{Shaded}
\begin{Highlighting}[]
\FunctionTok{as.factor}\NormalTok{(sexo)}
\end{Highlighting}
\end{Shaded}

\begin{verbatim}
[1] F F M M F
Levels: F M
\end{verbatim}

\noindent Observa-se que a linha adicional \texttt{Levels:\ F\ M}
indicam as categorias. Por padrão, o R ordena esses níveis em ordem
alfabética. Para facilitar os cálculos e análises, o R interpreta os
níveis categóricos como sendo números distintos, sendo assim, dentro do
nosso exemplo \texttt{F} representaria o número \texttt{0} e \texttt{M}
representaria o \texttt{1}.

\subsection{Valores especiais}\label{valores-especiais}

Valores como \texttt{NA}, \texttt{NaN}, \texttt{Inf} e \texttt{NULL}
ocorrem frequentemente dentro do mundo da programação estatística no R.
Em resumo:

\begin{itemize}
\tightlist
\item
  \texttt{NA} representa a Ausência de Informação. Suponha que o vetor
  \texttt{idades} que representa a idade de três pessoas. Uma situação
  que pode ocorrer é \texttt{idades\ \textless{}-\ c(10,\ NA,\ NA)}.
  Portanto, não é sabido a idade das pessoas 2 e 3.
\item
  \texttt{NaN} representa indefinições matemáticas. Um exemplo típico é
  o valor \(\log{-1}\), do qual \(x = -1\) não pertence aos possíveis
  valores de saída da função logarítmica, gerando um \texttt{NaN}
  (\emph{Not a number}).
\end{itemize}

\begin{Shaded}
\begin{Highlighting}[]
\FunctionTok{log}\NormalTok{(}\SpecialCharTok{{-}}\DecValTok{1}\NormalTok{)}
\end{Highlighting}
\end{Shaded}

\begin{verbatim}
Warning in log(-1): NaNs produzidos
\end{verbatim}

\begin{verbatim}
[1] NaN
\end{verbatim}

\begin{itemize}
\tightlist
\item
  \texttt{Inf} representa um número muito grande ou um limite
  matemático. Exemplos:
\end{itemize}

\begin{Shaded}
\begin{Highlighting}[]
\CommentTok{\# Número muito grande}
\DecValTok{10}\SpecialCharTok{\^{}}\DecValTok{510}
\end{Highlighting}
\end{Shaded}

\begin{verbatim}
[1] Inf
\end{verbatim}

\begin{Shaded}
\begin{Highlighting}[]
\CommentTok{\# Limite matemático}
\DecValTok{1}\SpecialCharTok{/}\DecValTok{0}
\end{Highlighting}
\end{Shaded}

\begin{verbatim}
[1] Inf
\end{verbatim}

\begin{itemize}
\tightlist
\item
  \texttt{NULL} representa a ausência de um objeto. Muitas vezes
  define-se um objeto como nulo para dizer ao R que não desejamos
  atribuir valores a ele.
\end{itemize}

\subsection{Pedindo ajuda}\label{pedindo-ajuda}

Uma das coisas que intimidam novos programadores, independente da
linguagem utilizada, é a ocorrência de erros. Neste sentido, o R pode
ser um grande aliado, pois ele relata mensagens, erros e avisos sobre o
código no console, como se fosse uma espécie de resposta e/ou
comunicação. As situações são:

\begin{itemize}
\tightlist
\item
  \texttt{Error}: em situações de erro legítimo aparecerá mensagens do
  tipo \texttt{Error\ in\ ...} e tentará explicar o que há de errado.
  Nestas situações o código, geralmente, não é executado. Por exemplo:
  \texttt{Error\ in\ ggplot(...)\ :\ could\ not\ find\ function\ "ggplot"}.
\item
  \texttt{Warning}: em situações de avisos, o R exibirá uma mensagem do
  tipo \texttt{Warning:\ ...} e tentará explicar o motivo do aviso.
  Geralmente, o código será executado, mas com algumas ressalvas. Por
  exemplo:
  \texttt{Warning:\ Removed\ 2\ rows\ containing\ missing\ values\ (geom\_point)}.
\item
  \texttt{Message}: quando o texto exibido não se enquadra nas duas
  opções anteriores, dizemos que é apenas uma mensagem. Pense, nessa
  situação, que tudo está acontecendo como o esperado e está tudo bem.
\end{itemize}

Quando surgir qualquer uma dessas saídas, não estaremos perdidos, pois o
R oferece mecanismos para encontrarmos respostas. Afinal, nem todo mundo
decorou todas as funções ou argumentos. Os principais mecanismos são:

\begin{itemize}
\tightlist
\item
  \texttt{?função} ou \texttt{help(função)} para consultar a
  documentação oficial.
\item
  \texttt{??termo} e \texttt{help.search("termo")} para buscas por
  palavras-chave.
\end{itemize}

Além disso, o RStudio oferece alguns
\href{https://rstudio.com/resources/cheatsheets/}{\emph{Cheatsheets}}
(resumo de códigos) que podem ajudar com determinados pacotes. E, por
fim, existem grandes comunidade online, tais como:
\href{https://stackoverflow.com/collectives/r-language}{Stack Overflow}
e \href{https://forum.posit.co/}{RStudio Community} dos quais também
podem serem úteis.

\chapter{Tidyverse}\label{tidyverse}

O \texttt{tidyverse} (\citeproc{ref-wickhamWelcomeTidyverse2019}{WICKHAM
ET AL., 2019}) é um ecossistema de pacotes R que reúne as tarefas
essenciais de qualquer fluxo de trabalho em ciência de dados:
importação, organização, manipulação, visualização e programação. Seu
principal objetivo é criar uma sintaxe consistente e legível,
facilitando a comunicação entre quem escreve o código e quem o executa.
Note-se que, embora o tidyverse cubra grande parte do fluxo de trabalho,
ele não inclui ferramentas específicas de modelagem estatística.

Para facilitar essa integração, o \texttt{tidyverse} utiliza
intensamente do operador pipe\footnote{A partir da versão 4.1 do R,
  existe também o operador pipe nativo
  \texttt{\textbar{}\textgreater{}}. No entanto, nesta apostila
  manteremos o uso de \texttt{\%\textgreater{}\%}, amplamente adotado no
  contexto do tidyverse.} (\texttt{\%\textgreater{}\%}) , que passa o
resultado de uma etapa diretamente para a próxima, evitando aninhamentos
confusos. Ao carregar o pacote, diversos módulos são automaticamente
disponibilizados:

\begin{Shaded}
\begin{Highlighting}[]
\FunctionTok{library}\NormalTok{(tidyverse)}
\end{Highlighting}
\end{Shaded}

\begin{Shaded}
\begin{Highlighting}[]
\NormalTok{{-}{-} Attaching core tidyverse packages {-}{-}{-}{-}{-}{-} tidyverse 2.0.0 {-}{-}}
\NormalTok{✔ dplyr     1.1.4     ✔ readr     2.1.5}
\NormalTok{✔ forcats   1.0.0     ✔ stringr   1.5.1}
\NormalTok{✔ ggplot2   3.5.2     ✔ tibble    3.2.1}
\NormalTok{✔ lubridate 1.9.4     ✔ tidyr     1.3.1}
\NormalTok{✔ purrr     1.0.4     }
\NormalTok{{-}{-} Conflicts {-}{-}{-}{-}{-}{-}{-}{-}{-}{-}{-}{-}{-}{-}{-}{-}{-}{-}{-}{-}{-}{-}{-}{-}{-}{-}{-}{-}{-}{-} tidyverse\_conflicts() {-}{-}}
\NormalTok{x dplyr::filter() masks stats::filter()}
\NormalTok{x dplyr::lag()    masks stats::lag()}
\NormalTok{i Use the conflicted package (http://conflicted.r{-}lib.org/) }
\NormalTok{  to force all conflicts to become errors}
\end{Highlighting}
\end{Shaded}

\noindent Entre os principais estão:

\begin{itemize}
\tightlist
\item
  \texttt{ggplot2} (visualização de dados);
\item
  \texttt{dplyr} (manipulação de dados);
\item
  \texttt{tidyr} (formatação ``\emph{long}''/``\emph{wide}'');
\item
  \texttt{readr} (leitura eficiente de arquivos de texto);
\item
  \texttt{tibble} (versão moderna do \texttt{data.frame});
\item
  \texttt{purrr} (programação funcional);
\item
  \texttt{stringr}, \texttt{forcats} e outros.
\end{itemize}

Como dito, muitos pacotes definem funções com nomes idênticos, sendo
comum que o console exiba nomes como:

\begin{verbatim}
The following objects are masked from ‘package:stats’:
    filter, lag
\end{verbatim}

Um pilar do \texttt{tidyverse} é a adoção do princípio \texttt{tidy}
(\citeproc{ref-wickhamTidyData2014}{WICKHAM, 2014}), em que:

\begin{itemize}
\tightlist
\item
  Cada variável ocupa uma coluna;
\item
  Cada observação ocupa uma linha;
\item
  Cada tipo de entidade observacional fica em sua própria tabela.
\end{itemize}

Nesse contexto, a \textbf{entidade observacional} é o conceito central
que define o que uma linha representa. Pode ser um paciente em um estudo
clínico, um país em dados econômicos ou, como nos exemplos a seguir:

\begin{itemize}
\tightlist
\item
  \textbf{Aves}: Cada linha corresponde a uma única ave, registrando
  suas características (peso, envergadura, espécie, etc.).
\item
  \textbf{Plantas}: Cada linha representa um vaso de planta em um
  experimento (altura, número de folhas, tipo de solo, etc.).
\end{itemize}

A estrutura de dados que implementa essa filosofia no \texttt{tidyverse}
é o \texttt{tibble}. Ele é a versão moderna do \texttt{data.frame},
projetado para ser mais prático e informativo, exibindo resumos concisos
dos dados e fornecendo diagnósticos mais úteis.

Uma vez apresentada a filosofia e a estrutura de dados do
\texttt{tidyverse}, o foco se volta para a aplicação prática. A seguir,
a concentração do material residirá nos dois pacotes centrais do
\texttt{tidyverse}: o \texttt{dplyr}, para manipulação de dados, e o
\texttt{ggplot2}, para a criação de gráficos.

\section{\texorpdfstring{Manipulação de dados com o pacote
\texttt{dplyr}}{Manipulação de dados com o pacote dplyr}}\label{manipulauxe7uxe3o-de-dados-com-o-pacote-dplyr}

O \texttt{dplyr} é um pacote do \texttt{tidyverse} que fornece um
conjunto de ferramentas robustas e intuitivas para manipulação de dados.
Os comandos oferecidos soam um tanto quanto intuitivos, correspondendo
ações comuns na área de análise de dados. Para explorar as principais
funções será utilizado o \emph{dataset} \texttt{penguins}, focando em
processos de filtragem, organização, transformação e resumos dos dados,
permitindo responder a perguntas básicas sobre a biologia e ecologia dos
pinguins.

O primeiro passo a ser feito é instalar a biblioteca
\texttt{palmerpenguins} e, em seguida, carregá-la no ambiente de
trabalho, para que possamos realizar uma inspeção inicial na estrutura
dos dados.

\begin{Shaded}
\begin{Highlighting}[]
\FunctionTok{install.packages}\NormalTok{(}\StringTok{"palmerpenguins"}\NormalTok{) }\CommentTok{\# Realizar apenas uma única vez}
\end{Highlighting}
\end{Shaded}

\begin{Shaded}
\begin{Highlighting}[]
\FunctionTok{library}\NormalTok{(palmerpenguins)}
\end{Highlighting}
\end{Shaded}

\noindent Para carregarmos os dados sobre pinguins no ambiente de
trabalho, podemos utilizar a função \texttt{data()}:

\begin{Shaded}
\begin{Highlighting}[]
\FunctionTok{data}\NormalTok{(}\StringTok{"penguins"}\NormalTok{, }\AttributeTok{package =} \StringTok{"palmerpenguins"}\NormalTok{)}
\end{Highlighting}
\end{Shaded}

Podemos observar que no painel \textbf{Environment} do RStudio, aparece
o objeto \texttt{penguins}, isso significa que o conjunto de dados está
carregado no ambiente de trabalho e podemos dar início nas inspeções. O
primeiro comando que será visto é o \texttt{glimpse()}. Ele exibe, de
maneira prática e rápida, a estrutura do \emph{dataset} como: dimensão
(número de linhas e colunas), o nome de cada coluna, o tipo de dado de
cada coluna e as primeiras observações.

\begin{Shaded}
\begin{Highlighting}[]
\FunctionTok{glimpse}\NormalTok{(penguins)}
\end{Highlighting}
\end{Shaded}

\begin{verbatim}
Rows: 344
Columns: 8
$ species           <fct> Adelie, Adelie, Adelie, Adelie, Adelie, Adelie, Adel~
$ island            <fct> Torgersen, Torgersen, Torgersen, Torgersen, Torgerse~
$ bill_length_mm    <dbl> 39.1, 39.5, 40.3, NA, 36.7, 39.3, 38.9, 39.2, 34.1, ~
$ bill_depth_mm     <dbl> 18.7, 17.4, 18.0, NA, 19.3, 20.6, 17.8, 19.6, 18.1, ~
$ flipper_length_mm <int> 181, 186, 195, NA, 193, 190, 181, 195, 193, 190, 186~
$ body_mass_g       <int> 3750, 3800, 3250, NA, 3450, 3650, 3625, 4675, 3475, ~
$ sex               <fct> male, female, female, NA, female, male, female, male~
$ year              <int> 2007, 2007, 2007, 2007, 2007, 2007, 2007, 2007, 2007~
\end{verbatim}

A saída deste comando revela que existem 344 observações e 8 variáveis,
sendo elas \texttt{species}, \texttt{island}, \texttt{bill\_length\_mm},
\texttt{flipper\_length\_mm}, \texttt{body\_mass\_g}, \texttt{sex} e
\texttt{year}, com seus respectivos tipos, como \texttt{factor} para
\texttt{species} e \texttt{numeric} para \texttt{bill\_length\_mm}. Além
disso, é possível observar dados ausentes em algumas variáveis,
representados por \texttt{NA}. Em geral, nos \emph{datasets} disponíveis
em pacotes R, é possível utilizar o comando \texttt{help(penguins)} para
buscar informações sobre o conjunto de dados que será trabalhado.

Executando o comando de ajuda, são obtidas as seguintes informações
sobre as variáveis:

\begin{itemize}
\tightlist
\item
  \texttt{species}: um fator que denota a espécie do pinguim (Adélie,
  Chinstrap ou Gentoo).
\item
  \texttt{island}: um fator que denota ilhas no Arquipélago Palmer na
  Antártica (Biscoe, Dream ou Torgersen).
\item
  \texttt{bill\_length\_mm}: um número que representa o comprimento do
  bico (em milímetros).
\item
  \texttt{bill\_depth\_mm}: um número que representa a profundidade do
  bico (em milímetros).
\item
  \texttt{flipper\_length\_mm}: um número inteiro que representa o
  comprimento da nadadeira (em milímetros).
\item
  \texttt{body\_mass\_g}: um número inteiro que representa a massa do
  animal (em gramas).
\item
  \texttt{sex}: um fator que representa o sexo do animal (feminino ou
  masculino).
\item
  \texttt{year}: um número inteiro que denota o ano de estudo (2007,
  2008 ou 2009).
\end{itemize}

Adicionalmente, também é informado que os dados foram originalmente
publicados no estudo de Gorman et al.
(\citeproc{ref-gormanEcologicalSexualDimorphism2014}{2014}) e que essa
pesquisa fez parte do programa \emph{Palmer Station Long-Term Ecological
Research} (LTER). Isso significa que o conjunto de dados que está sendo
utilizado possui uma origem científica real, ligada a questões sobre
como o ambiente e as diferenças entre sexos afetam a vida dessas aves.

A segunda função que será vista é o \texttt{select()}. Frequentemente,
um conjunto de dados contém mais informações do que o necessário para
uma análise específica. Com isso em mente, a função \texttt{select()}
permite-nos selecionar colunas de interesse. Em geral, os argumentos são
os nomes das colunas.

\begin{Shaded}
\begin{Highlighting}[]
\NormalTok{penguins }\SpecialCharTok{\%\textgreater{}\%} 
  \FunctionTok{select}\NormalTok{(species, island, sex)}
\end{Highlighting}
\end{Shaded}

\begin{verbatim}
# A tibble: 344 x 3
   species island    sex   
   <fct>   <fct>     <fct> 
 1 Adelie  Torgersen male  
 2 Adelie  Torgersen female
 3 Adelie  Torgersen female
 4 Adelie  Torgersen <NA>  
 5 Adelie  Torgersen female
 6 Adelie  Torgersen male  
 7 Adelie  Torgersen female
 8 Adelie  Torgersen male  
 9 Adelie  Torgersen <NA>  
10 Adelie  Torgersen <NA>  
# i 334 more rows
\end{verbatim}

\noindent O \texttt{dplyr} também oferece ``seletores auxiliares'' que
tornam a seleção mais poderosa e flexível. Por exemplo, caso desejarmos
selecionar todas as medidas biométricas contidas no \emph{dataset} que
terminam com \texttt{\_mm}, é possível usar a função-argumento
\texttt{ends\_with()} dentro de \texttt{select()}:

\begin{Shaded}
\begin{Highlighting}[]
\NormalTok{penguins }\SpecialCharTok{\%\textgreater{}\%} 
  \FunctionTok{select}\NormalTok{(}
\NormalTok{    body\_mass\_g, }\FunctionTok{ends\_with}\NormalTok{(}\StringTok{"\_mm"}\NormalTok{)}
\NormalTok{  )}
\end{Highlighting}
\end{Shaded}

\begin{verbatim}
# A tibble: 344 x 4
   body_mass_g bill_length_mm bill_depth_mm flipper_length_mm
         <int>          <dbl>         <dbl>             <int>
 1        3750           39.1          18.7               181
 2        3800           39.5          17.4               186
 3        3250           40.3          18                 195
 4          NA           NA            NA                  NA
 5        3450           36.7          19.3               193
 6        3650           39.3          20.6               190
 7        3625           38.9          17.8               181
 8        4675           39.2          19.6               195
 9        3475           34.1          18.1               193
10        4250           42            20.2               190
# i 334 more rows
\end{verbatim}

\noindent Outros seletores úteis incluem \texttt{starts\_with()} e
\texttt{contains()}. Para remover colunas, utiliza-se o sinal de menos
(\texttt{-}). Por exemplo, deseja-se remover as colunas \texttt{ano} e
\texttt{island}:

\begin{Shaded}
\begin{Highlighting}[]
\NormalTok{penguins }\SpecialCharTok{\%\textgreater{}\%} 
  \FunctionTok{select}\NormalTok{(}\SpecialCharTok{{-}}\NormalTok{year, }\SpecialCharTok{{-}}\NormalTok{island)}
\end{Highlighting}
\end{Shaded}

\begin{verbatim}
# A tibble: 344 x 6
   species bill_length_mm bill_depth_mm flipper_length_mm body_mass_g sex   
   <fct>            <dbl>         <dbl>             <int>       <int> <fct> 
 1 Adelie            39.1          18.7               181        3750 male  
 2 Adelie            39.5          17.4               186        3800 female
 3 Adelie            40.3          18                 195        3250 female
 4 Adelie            NA            NA                  NA          NA <NA>  
 5 Adelie            36.7          19.3               193        3450 female
 6 Adelie            39.3          20.6               190        3650 male  
 7 Adelie            38.9          17.8               181        3625 female
 8 Adelie            39.2          19.6               195        4675 male  
 9 Adelie            34.1          18.1               193        3475 <NA>  
10 Adelie            42            20.2               190        4250 <NA>  
# i 334 more rows
\end{verbatim}

Antes prosseguirmos para a próxima função, vale destacar que o conjunto
de dados \texttt{penguins} é um objeto \texttt{tibble} dentro do R e,
portanto, por mais que existam 344 observações, o \texttt{tibble} enxuga
a visualização para somente 10, além de indicar quantas linhas ainda
existem.

A terceira função é o \texttt{filter()}. Enquanto \texttt{select()}
trabalha nas colunas, o \texttt{filter()} trabalha nas linhas,
permitindo-nos manter apenas as observações que satisfazem certas
condições. É aqui que é possível responder perguntas investigadas com
relação aos dados. Por exemplo, para encontrar todos os pinguins da
espécie Adelie que vivem na ilha Torgersen:

\begin{Shaded}
\begin{Highlighting}[]
\NormalTok{penguins }\SpecialCharTok{\%\textgreater{}\%} 
  \FunctionTok{filter}\NormalTok{(}
\NormalTok{      species }\SpecialCharTok{==} \StringTok{"Adelie"}\NormalTok{, island }\SpecialCharTok{==} \StringTok{"Torgersen"}
\NormalTok{    )}
\end{Highlighting}
\end{Shaded}

\begin{verbatim}
# A tibble: 52 x 8
   species island    bill_length_mm bill_depth_mm flipper_length_mm body_mass_g
   <fct>   <fct>              <dbl>         <dbl>             <int>       <int>
 1 Adelie  Torgersen           39.1          18.7               181        3750
 2 Adelie  Torgersen           39.5          17.4               186        3800
 3 Adelie  Torgersen           40.3          18                 195        3250
 4 Adelie  Torgersen           NA            NA                  NA          NA
 5 Adelie  Torgersen           36.7          19.3               193        3450
 6 Adelie  Torgersen           39.3          20.6               190        3650
 7 Adelie  Torgersen           38.9          17.8               181        3625
 8 Adelie  Torgersen           39.2          19.6               195        4675
 9 Adelie  Torgersen           34.1          18.1               193        3475
10 Adelie  Torgersen           42            20.2               190        4250
# i 42 more rows
# i 2 more variables: sex <fct>, year <int>
\end{verbatim}

Neste exemplo, as condições separadas por vírgula são unidas por um
``E'' lógico. Também é possível utilizar o ``OU'' lógico para determinar
pinguins mais pesados (acima de 6000g) ou com bicos muito longos (mais
de 55mm) através do conectivo \texttt{\textbar{}}:

\begin{Shaded}
\begin{Highlighting}[]
\NormalTok{penguins }\SpecialCharTok{\%\textgreater{}\%} 
  \FunctionTok{filter}\NormalTok{(}
\NormalTok{    body\_mass\_g }\SpecialCharTok{\textgreater{}} \DecValTok{6000} \SpecialCharTok{|}\NormalTok{ bill\_length\_mm }\SpecialCharTok{\textgreater{}} \DecValTok{55}
\NormalTok{  )}
\end{Highlighting}
\end{Shaded}

\begin{verbatim}
# A tibble: 6 x 8
  species   island bill_length_mm bill_depth_mm flipper_length_mm body_mass_g
  <fct>     <fct>           <dbl>         <dbl>             <int>       <int>
1 Gentoo    Biscoe           49.2          15.2               221        6300
2 Gentoo    Biscoe           59.6          17                 230        6050
3 Gentoo    Biscoe           55.9          17                 228        5600
4 Gentoo    Biscoe           55.1          16                 230        5850
5 Chinstrap Dream            58            17.8               181        3700
6 Chinstrap Dream            55.8          19.8               207        4000
# i 2 more variables: sex <fct>, year <int>
\end{verbatim}

O \texttt{filter()} também permite encontrar valores ausentes
(\texttt{NA}s) em conjunto da função \texttt{is.na()}. Por exemplo,
deseja-se verificar quais pinguins não tiveram seu sexo registrado:

\begin{Shaded}
\begin{Highlighting}[]
\NormalTok{penguins }\SpecialCharTok{\%\textgreater{}\%} 
  \FunctionTok{filter}\NormalTok{(}\FunctionTok{is.na}\NormalTok{(sex))}
\end{Highlighting}
\end{Shaded}

\begin{verbatim}
# A tibble: 11 x 8
   species island    bill_length_mm bill_depth_mm flipper_length_mm body_mass_g
   <fct>   <fct>              <dbl>         <dbl>             <int>       <int>
 1 Adelie  Torgersen           NA            NA                  NA          NA
 2 Adelie  Torgersen           34.1          18.1               193        3475
 3 Adelie  Torgersen           42            20.2               190        4250
 4 Adelie  Torgersen           37.8          17.1               186        3300
 5 Adelie  Torgersen           37.8          17.3               180        3700
 6 Adelie  Dream               37.5          18.9               179        2975
 7 Gentoo  Biscoe              44.5          14.3               216        4100
 8 Gentoo  Biscoe              46.2          14.4               214        4650
 9 Gentoo  Biscoe              47.3          13.8               216        4725
10 Gentoo  Biscoe              44.5          15.7               217        4875
11 Gentoo  Biscoe              NA            NA                  NA          NA
# i 2 more variables: sex <fct>, year <int>
\end{verbatim}

A interpretação do \texttt{NA} é relativa ao contexto dos dados. No caso
das observações sobre os pinguins, os valores ausentes na variável
\texttt{sex} permite identificar pinguins que não tiveram o sexo
avaliado, tornando um provável erro frustrante de coleta de dados para
um objeto de investigação. O pacote \texttt{tidyr}, também do
\texttt{tidyverse}, oferece a função \texttt{drop\_na()}, que remove
quaisquer linhas que contenham \texttt{NA}s, permitindo a criação de um
\emph{dataset} auxiliar:

\begin{Shaded}
\begin{Highlighting}[]
\NormalTok{penguins\_completos }\OtherTok{\textless{}{-}}\NormalTok{ penguins }\SpecialCharTok{\%\textgreater{}\%} 
  \FunctionTok{drop\_na}\NormalTok{()}
\end{Highlighting}
\end{Shaded}

A quarta função que será apresentada é \texttt{arrange()}, que permite
reordenar as linhas do dataframe com base nos valores de uma ou mais
colunas. Isso é útil para encontrar extremos ou simplesmente para
organizar a saída de uma forma mais lógica. Para encontrar os pinguins
mais leves, ordenamos pela massa corporal em ordem crescente (o padrão):

\begin{Shaded}
\begin{Highlighting}[]
\NormalTok{penguins }\SpecialCharTok{\%\textgreater{}\%} 
  \FunctionTok{arrange}\NormalTok{(body\_mass\_g)}
\end{Highlighting}
\end{Shaded}

\begin{verbatim}
# A tibble: 344 x 8
   species   island   bill_length_mm bill_depth_mm flipper_length_mm body_mass_g
   <fct>     <fct>             <dbl>         <dbl>             <int>       <int>
 1 Chinstrap Dream              46.9          16.6               192        2700
 2 Adelie    Biscoe             36.5          16.6               181        2850
 3 Adelie    Biscoe             36.4          17.1               184        2850
 4 Adelie    Biscoe             34.5          18.1               187        2900
 5 Adelie    Dream              33.1          16.1               178        2900
 6 Adelie    Torgers~           38.6          17                 188        2900
 7 Chinstrap Dream              43.2          16.6               187        2900
 8 Adelie    Biscoe             37.9          18.6               193        2925
 9 Adelie    Dream              37.5          18.9               179        2975
10 Adelie    Dream              37            16.9               185        3000
# i 334 more rows
# i 2 more variables: sex <fct>, year <int>
\end{verbatim}

\noindent Para ordenar os valores em ordem decrescente (do maior para o
menor), utilizamos a função auxiliar \texttt{desc()}, desta maneira,
encontramos os pinguins mais pesados:

\begin{Shaded}
\begin{Highlighting}[]
\NormalTok{penguins }\SpecialCharTok{\%\textgreater{}\%} 
  \FunctionTok{arrange}\NormalTok{(}\FunctionTok{desc}\NormalTok{(body\_mass\_g))}
\end{Highlighting}
\end{Shaded}

\begin{verbatim}
# A tibble: 344 x 8
   species island bill_length_mm bill_depth_mm flipper_length_mm body_mass_g
   <fct>   <fct>           <dbl>         <dbl>             <int>       <int>
 1 Gentoo  Biscoe           49.2          15.2               221        6300
 2 Gentoo  Biscoe           59.6          17                 230        6050
 3 Gentoo  Biscoe           51.1          16.3               220        6000
 4 Gentoo  Biscoe           48.8          16.2               222        6000
 5 Gentoo  Biscoe           45.2          16.4               223        5950
 6 Gentoo  Biscoe           49.8          15.9               229        5950
 7 Gentoo  Biscoe           48.4          14.6               213        5850
 8 Gentoo  Biscoe           49.3          15.7               217        5850
 9 Gentoo  Biscoe           55.1          16                 230        5850
10 Gentoo  Biscoe           49.5          16.2               229        5800
# i 334 more rows
# i 2 more variables: sex <fct>, year <int>
\end{verbatim}

\noindent Também é possível ordenar múltiplas colunas. Por exemplo, para
encontrar o pinguim mais pesado dentro de cada espécie:

\begin{Shaded}
\begin{Highlighting}[]
\NormalTok{penguins }\SpecialCharTok{\%\textgreater{}\%} 
  \FunctionTok{arrange}\NormalTok{(}
\NormalTok{    species, }\CommentTok{\# Primeiro por espécie}
    \FunctionTok{desc}\NormalTok{(body\_mass\_g) }\CommentTok{\# Depois por massa decrescente}
\NormalTok{  )}
\end{Highlighting}
\end{Shaded}

\begin{verbatim}
# A tibble: 344 x 8
   species island    bill_length_mm bill_depth_mm flipper_length_mm body_mass_g
   <fct>   <fct>              <dbl>         <dbl>             <int>       <int>
 1 Adelie  Biscoe              43.2          19                 197        4775
 2 Adelie  Biscoe              41            20                 203        4725
 3 Adelie  Torgersen           42.9          17.6               196        4700
 4 Adelie  Torgersen           39.2          19.6               195        4675
 5 Adelie  Dream               39.8          19.1               184        4650
 6 Adelie  Dream               39.6          18.8               190        4600
 7 Adelie  Biscoe              45.6          20.3               191        4600
 8 Adelie  Torgersen           42.5          20.7               197        4500
 9 Adelie  Dream               37.5          18.5               199        4475
10 Adelie  Torgersen           41.8          19.4               198        4450
# i 334 more rows
# i 2 more variables: sex <fct>, year <int>
\end{verbatim}

A quinta função e, com certeza, uma das mais funcionais é a
\texttt{mutate()}. Ela permite criar novas colunas (variáveis) que são
funções de colunas já existentes, sem modificar as originais. Por
exemplo, suponha que desejamos mostrar somente as espécies e massas de
pinguins em quilogramas (kg):

\begin{Shaded}
\begin{Highlighting}[]
\NormalTok{penguins }\SpecialCharTok{\%\textgreater{}\%} 
  \FunctionTok{mutate}\NormalTok{(}\AttributeTok{body\_mass\_kg =}\NormalTok{ body\_mass\_g}\SpecialCharTok{/}\DecValTok{1000}\NormalTok{) }\SpecialCharTok{\%\textgreater{}\%} 
  \FunctionTok{select}\NormalTok{(species, body\_mass\_kg)}
\end{Highlighting}
\end{Shaded}

\begin{verbatim}
# A tibble: 344 x 2
   species body_mass_kg
   <fct>          <dbl>
 1 Adelie          3.75
 2 Adelie          3.8 
 3 Adelie          3.25
 4 Adelie         NA   
 5 Adelie          3.45
 6 Adelie          3.65
 7 Adelie          3.62
 8 Adelie          4.68
 9 Adelie          3.48
10 Adelie          4.25
# i 334 more rows
\end{verbatim}

\noindent Podemos usar \texttt{mutate()} para criar categorias. A função
\texttt{case\_when()} é extremamente útil para criar classificações
baseadas em condições lógicas., Suponha que desejamos criar uma
categoria de tamanho baseada na massa corporal:

\begin{Shaded}
\begin{Highlighting}[]
\NormalTok{penguins }\SpecialCharTok{\%\textgreater{}\%} 
  \FunctionTok{mutate}\NormalTok{(}
    \AttributeTok{size\_category =} \FunctionTok{case\_when}\NormalTok{(}
\NormalTok{      body\_mass\_g }\SpecialCharTok{\textgreater{}} \DecValTok{4750} \SpecialCharTok{\textasciitilde{}} \StringTok{"Grande"}\NormalTok{,}
\NormalTok{      body\_mass\_g }\SpecialCharTok{\textless{}} \DecValTok{3500} \SpecialCharTok{\textasciitilde{}} \StringTok{"Pequeno"}\NormalTok{,}
      \ConstantTok{TRUE} \SpecialCharTok{\textasciitilde{}} \StringTok{"Médio"}
\NormalTok{    )}
\NormalTok{  ) }\SpecialCharTok{\%\textgreater{}\%} 
  \FunctionTok{select}\NormalTok{(}
\NormalTok{    species, body\_mass\_g, size\_category}
\NormalTok{  )}
\end{Highlighting}
\end{Shaded}

\begin{verbatim}
# A tibble: 344 x 3
   species body_mass_g size_category
   <fct>         <int> <chr>        
 1 Adelie         3750 Médio        
 2 Adelie         3800 Médio        
 3 Adelie         3250 Pequeno      
 4 Adelie           NA Médio        
 5 Adelie         3450 Pequeno      
 6 Adelie         3650 Médio        
 7 Adelie         3625 Médio        
 8 Adelie         4675 Médio        
 9 Adelie         3475 Pequeno      
10 Adelie         4250 Médio        
# i 334 more rows
\end{verbatim}

As funções \texttt{group\_by()} e \texttt{summarise()} formam uma dupla
formidável para agrupar e resumir os dados, pertencendo ao coração da
análise de dados. A função \texttt{summarise()} erve para calcular
estatísticas resumidas (como média, total, mínimo etc.) e, quando usada
em conjunto com \texttt{group\_by()} permite gerar resumos por grupo.

Inicialmente, vamos utilizar o \texttt{summarise()} no \emph{dataset}
completo para obter estatísticas globais. Não obstante, é bom frisar a
utilização do argumento \texttt{na.rm\ =\ TRUE} para instruir a remoção
dos valores \texttt{NA}.

\begin{Shaded}
\begin{Highlighting}[]
\NormalTok{penguins }\SpecialCharTok{\%\textgreater{}\%} 
  \FunctionTok{summarise}\NormalTok{(}
    \AttributeTok{massa\_media =} \FunctionTok{mean}\NormalTok{(body\_mass\_g, }\AttributeTok{na.rm =} \ConstantTok{TRUE}\NormalTok{),}
    \AttributeTok{nadadeira\_max =} \FunctionTok{max}\NormalTok{(flipper\_length\_mm, }\AttributeTok{na.rm =} \ConstantTok{TRUE}\NormalTok{)}
\NormalTok{  )}
\end{Highlighting}
\end{Shaded}

\begin{verbatim}
# A tibble: 1 x 2
  massa_media nadadeira_max
        <dbl>         <int>
1       4202.           231
\end{verbatim}

No entanto, essas métricas não fornecem informações com relação as
espécies de pinguins. Para resolver isso e possibilitar que mais
perguntas sejam respondidas, a função \texttt{group\_by()} permite que o
R faça operações em subconjuntos. Por exemplo, suponha que desejamos
determinar qual é a massa corporal por espécie:

\begin{Shaded}
\begin{Highlighting}[]
\NormalTok{penguins }\SpecialCharTok{\%\textgreater{}\%} 
  \FunctionTok{group\_by}\NormalTok{(species) }\SpecialCharTok{\%\textgreater{}\%} 
  \FunctionTok{summarise}\NormalTok{(}
    \AttributeTok{massa\_media\_g =} \FunctionTok{mean}\NormalTok{(body\_mass\_g, }\AttributeTok{na.rm =} \ConstantTok{TRUE}\NormalTok{)}
\NormalTok{  )}
\end{Highlighting}
\end{Shaded}

\begin{verbatim}
# A tibble: 3 x 2
  species   massa_media_g
  <fct>             <dbl>
1 Adelie            3701.
2 Chinstrap         3733.
3 Gentoo            5076.
\end{verbatim}

\noindent Podemos fazer agrupamentos por múltiplas variáveis para
investigações mais profundas. Por exemplo, considere que um pesquisador
deseja explorar o dimorfismo sexual. Para isso, estatísticas por espécie
e sexo serão calculadas.

\begin{Shaded}
\begin{Highlighting}[]
\NormalTok{tabela\_resumo }\OtherTok{\textless{}{-}}\NormalTok{ penguins }\SpecialCharTok{\%\textgreater{}\%} 
  \FunctionTok{drop\_na}\NormalTok{(sex) }\SpecialCharTok{\%\textgreater{}\%} 
  \FunctionTok{group\_by}\NormalTok{(species, sex) }\SpecialCharTok{\%\textgreater{}\%} 
  \FunctionTok{summarise}\NormalTok{(}
    \AttributeTok{contagem =} \FunctionTok{n}\NormalTok{(),}
    \AttributeTok{massa\_media\_g =} \FunctionTok{mean}\NormalTok{(body\_mass\_g),}
    \AttributeTok{massa\_dp\_g =} \FunctionTok{sd}\NormalTok{(body\_mass\_g),}
    \AttributeTok{comp\_bico\_medio\_mm =} \FunctionTok{mean}\NormalTok{(bill\_length\_mm),}
    \AttributeTok{.groups =} \StringTok{"drop"}
\NormalTok{  )}
\NormalTok{tabela\_resumo}
\end{Highlighting}
\end{Shaded}

\begin{table}[H]

\caption{\label{tbl-tabela-resumo1}Estatísticas descritivas de
características biométricas de pinguins, agrupadas por espécie e sexo.}

\centering{

\centering
\resizebox{\ifdim\width>\linewidth\linewidth\else\width\fi}{!}{
\begin{tabular}{llcccc}
\toprule
Espécies & Sexo & Contagem & Massa média (g) & Massa Desvio-padrão (g) & Comprimento médio do bico (mm)\\
\midrule
\cellcolor{gray!10}{Adelie} & \cellcolor{gray!10}{female} & \cellcolor{gray!10}{73} & \cellcolor{gray!10}{3368.84} & \cellcolor{gray!10}{269.38} & \cellcolor{gray!10}{37.26}\\
Adelie & male & 73 & 4043.49 & 346.81 & 40.39\\
\cellcolor{gray!10}{Chinstrap} & \cellcolor{gray!10}{female} & \cellcolor{gray!10}{34} & \cellcolor{gray!10}{3527.21} & \cellcolor{gray!10}{285.33} & \cellcolor{gray!10}{46.57}\\
Chinstrap & male & 34 & 3938.97 & 362.14 & 51.09\\
\cellcolor{gray!10}{Gentoo} & \cellcolor{gray!10}{female} & \cellcolor{gray!10}{58} & \cellcolor{gray!10}{4679.74} & \cellcolor{gray!10}{281.58} & \cellcolor{gray!10}{45.56}\\
\addlinespace
Gentoo & male & 61 & 5484.84 & 313.16 & 49.47\\
\bottomrule
\end{tabular}}

}

\end{table}%

\noindent Vale reforçar que a Tabela~\ref{tbl-tabela-resumo1} foi gerada
usando o \texttt{dplyr}, com as funções auxiliares \texttt{n()} para
realizar a contagem de observações em cada grupo e
\texttt{drop\_na(sex)} para remover as observações onde o sexo é
desconhecido, permitindo avaliar dimorfismo sexual em todas as três
espécies, especialmente na massa corporal. O grande potencial dessa
tabela é obter respostas como:

\begin{itemize}
\tightlist
\item
  Os pinguins Gentoo são, em média, os mais pesados.
\item
  Dentro de cada espécie, os machos são consistentemente mais pesados e
  têm bicos mais longo que as fêmeas.
\end{itemize}

\noindent Esses resultados permitem tirar conclusões sobre algumas
hipóteses biológicas.

Por fim, a última função que será abordada é a \texttt{recode()}. Muitas
vezes, os nomes das categorias nos conjuntos de dados não são ideais
para a análise ou apresentação em gráficos. Podem ser longos demais,
estarem em outro idioma ou simplesmente não serem claros. Para isso, a
função \texttt{recode()} permite renomear valores de uma variável
categórica de forma simples e direta. Por exemplo, suponha que desejamos
traduzir os termos da variável \texttt{sex} da
Tabela~\ref{tbl-tabela-resumo1} para o português:

\begin{Shaded}
\begin{Highlighting}[]
\NormalTok{tabela\_resumo }\SpecialCharTok{\%\textgreater{}\%} 
  \FunctionTok{mutate}\NormalTok{(}
    \AttributeTok{sex =} \FunctionTok{recode}\NormalTok{(sex,}
                 \StringTok{"female"} \OtherTok{=} \StringTok{"Fêmea"}\NormalTok{,}
                 \StringTok{"male"} \OtherTok{=} \StringTok{"Macho"}\NormalTok{)}
\NormalTok{  ) }
\end{Highlighting}
\end{Shaded}

\begin{table}[H]

\caption{\label{tbl-traduzida}Tradução da variável \texttt{sexo} da
Tabela~\ref{tbl-tabela-resumo1}.}

\centering{

\centering
\resizebox{\ifdim\width>\linewidth\linewidth\else\width\fi}{!}{
\begin{tabular}{llcccc}
\toprule
Espécies & Sexo & Contagem & Massa média (g) & Massa Desvio-padrão (g) & Comprimento médio do bico (mm)\\
\midrule
\cellcolor{gray!10}{Adelie} & \cellcolor{gray!10}{Fêmea} & \cellcolor{gray!10}{73} & \cellcolor{gray!10}{3368.84} & \cellcolor{gray!10}{269.38} & \cellcolor{gray!10}{37.26}\\
Adelie & Macho & 73 & 4043.49 & 346.81 & 40.39\\
\cellcolor{gray!10}{Chinstrap} & \cellcolor{gray!10}{Fêmea} & \cellcolor{gray!10}{34} & \cellcolor{gray!10}{3527.21} & \cellcolor{gray!10}{285.33} & \cellcolor{gray!10}{46.57}\\
Chinstrap & Macho & 34 & 3938.97 & 362.14 & 51.09\\
\cellcolor{gray!10}{Gentoo} & \cellcolor{gray!10}{Fêmea} & \cellcolor{gray!10}{58} & \cellcolor{gray!10}{4679.74} & \cellcolor{gray!10}{281.58} & \cellcolor{gray!10}{45.56}\\
\addlinespace
Gentoo & Macho & 61 & 5484.84 & 313.16 & 49.47\\
\bottomrule
\end{tabular}}

}

\end{table}%

As principais funções do pacote \texttt{dplyr} que foram vistas estão
resumidas e descritas na Tabela \ref{tbl-dplyr-fcts} e agora que
aprendemos como manipular os dados com o \texttt{dplyr}, podemos avançar
para a construção de gráficos com o pacote \texttt{ggplot2}.

\begin{longtable}[t]{ll}

\caption{\label{tbl-dplyr-fcts}Descrição das principais funções do
\texttt{dplyr}.}

\tabularnewline

\toprule
Função & Descrição\\
\midrule
\cellcolor{gray!10}{glimpse()} & \cellcolor{gray!10}{Inspecionar conjuntos de dados.}\\
select() & Seleciona colunas pelo nome.\\
\cellcolor{gray!10}{filter()} & \cellcolor{gray!10}{Filtra linhas com base em seus valores.}\\
arrange() & Reordena as linhas.\\
\cellcolor{gray!10}{mutate()} & \cellcolor{gray!10}{Cria novas colunas (variáveis).}\\
\addlinespace
group\_by() & Agrupa os dados por uma ou mais variáveis.\\
\cellcolor{gray!10}{summarise()} & \cellcolor{gray!10}{Reduz múltiplos valores a um único resumo.}\\
recode() & Renomeia categorias de variáveis.\\
\cellcolor{gray!10}{n()} & \cellcolor{gray!10}{Conta o número de observações.}\\
\bottomrule

\end{longtable}

\section{\texorpdfstring{A Arte da Visualização de Dados:
\texttt{ggplot2}}{A Arte da Visualização de Dados: ggplot2}}\label{a-arte-da-visualizauxe7uxe3o-de-dados-ggplot2}

\part{Fundamentos do Pensamento Estatístico}

\chapter{Princípios de
Probabilidade}\label{princuxedpios-de-probabilidade}

texto teste \(f(x) = ax^2\).

\chapter{A Lógica da Inferência
Estatística}\label{a-luxf3gica-da-inferuxeancia-estatuxedstica}

Sejam os testes de hipóteses \(H_0\) \emph{vs}. \(H_A\).

\chapter{Delineamento de
Experimentos}\label{delineamento-de-experimentos}

Um delineamento é\ldots{}

\part{Modelagem Estatística de Dados Biológicos}

\chapter{Modelos Lineares -- A Base da
Modelagem}\label{modelos-lineares-a-base-da-modelagem}

Um modelo linear é definido por: \[Y = X\beta + \epsilon\]

\chapter{Entendendo os Modelos
Mistos}\label{entendendo-os-modelos-mistos}

Os efeitos aleatórios são\ldots{}

\chapter{\texorpdfstring{Modelos Lineares Mistos com
\texttt{lme4}}{Modelos Lineares Mistos com }}\label{modelos-lineares-mistos-com}

O pacote \texttt{lme4} nos permite modelar\ldots{}

\chapter{Modelos Lineares Generalizados
Mistos}\label{modelos-lineares-generalizados-mistos}

Um GLMM é definino como sendo\ldots{}

\chapter{Validação e Interpretação de Modelos
Mistos}\label{validauxe7uxe3o-e-interpretauxe7uxe3o-de-modelos-mistos}

Após ajustar os modelos, é boa prática validá-los através de
técnicas\ldots{}

\part{Aplicações Práticas e Recursos}

\bookmarksetup{startatroot}

\chapter*{Referências}\label{referuxeancias}
\addcontentsline{toc}{chapter}{Referências}

\markboth{Referências}{Referências}

\phantomsection\label{refs}
\begin{CSLReferences}{0}{1}
\bibitem[\citeproctext]{ref-gormanEcologicalSexualDimorphism2014}
GORMAN, Kristen B. \emph{et al.}
\href{https://doi.org/10.1371/journal.pone.0090081}{Ecological {Sexual
Dimorphism} and {Environmental Variability} within a {Community} of
{Antarctic Penguins} ({Genus Pygoscelis})}. \textbf{PLOS ONE}, v. 9, n.
3, p. e90081, 2014.

\bibitem[\citeproctext]{ref-wickhamTidyData2014}
WICKHAM, Hadley. \href{https://doi.org/10.18637/jss.v059.i10}{Tidy
{Data}}. \textbf{Journal of Statistical Software}, v. 59, p. 1--23, set.
2014.

\bibitem[\citeproctext]{ref-wickhamWelcomeTidyverse2019}
WICKHAM, Hadley \emph{et al.}
\href{https://doi.org/10.21105/joss.01686}{Welcome to the {Tidyverse}}.
\textbf{Journal of Open Source Software}, v. 4, n. 43, p. 1686, nov.
2019.

\bibitem[\citeproctext]{ref-zuurAnalysingEcologicalData2007}
ZUUR, Alain F. \emph{et al.}
\textbf{\href{https://doi.org/10.1007/978-0-387-45972-1}{Analysing
{Ecological Data}}}. New York, NY: Springer, 2007.

\end{CSLReferences}




\end{document}
