% header.tex

% -------------------------------------------------
% 1. Layout geral e margens
% -------------------------------------------------
\usepackage{geometry}
\geometry{margin=1in}
\usepackage{float}
\usepackage[shortlabels]{enumitem}

% -------------------------------------------------
% 2. Matemática e teoremas
% -------------------------------------------------
\usepackage{amsmath, amsthm, amssymb}
\usepackage[framemethod=tikz]{mdframed}

% -------------------------------------------------
% 2.1. Desenhos
% -------------------------------------------------

\usepackage{pgfplotstable}
\usepackage{pgfplots}
  \pgfplotsset{compat=1.17}
  \usepgfplotslibrary{statistics}
  \usepgfplotslibrary{colorbrewer}
\usepackage{tikz}
  \usetikzlibrary{intersections}
  \usetikzlibrary{patterns}
  \usetikzlibrary{arrows,fit,shapes.geometric}
  \usetikzlibrary{pgfplots.statistics, pgfplots.colorbrewer}
\usepackage{tkz-euclide}

% -------------------------------------------------
% 2.2. Ambientes
% -------------------------------------------------
\theoremstyle{definition}
\newtheorem{questao}{Questão}[section]
\renewcommand{\thequestao}{\arabic{questao}}
\newtheorem*{solucao}{Solução}
\newtheorem*{exemplo}{Exemplo}
\theoremstyle{exemplo}
\newmdtheoremenv[
  hidealllines=true,
  leftline=true,
  linewidth=3pt,
  innerleftmargin=10pt,
  innerrightmargin=10pt,
  innertopmargin=0pt,
  linecolor=black!10,
]{exemple}{Exemplo}

% -------------------------------------------------
% 3. Fonte, espaçamento e microtipografia
% -------------------------------------------------
\usepackage{lmodern}
\usepackage{textcomp}
\usepackage{microtype}
\usepackage{setspace}
\onehalfspacing

% -------------------------------------------------
% 4. Primeiro parágrafo e identação
% -------------------------------------------------
\usepackage{indentfirst}
\setlength{\parindent}{1.25cm}
\setlength{\parskip}{0pt}

% -------------------------------------------------
% 5. Sumário (toc) customizado
% -------------------------------------------------
\usepackage{tocloft}
\renewcommand{\cftpartleader}{\cftdotfill{\cftdotsep}}
\renewcommand{\cftchapleader}{\cftdotfill{\cftdotsep}}

% -------------------------------------------------
% 6. Tabelas
% -------------------------------------------------
\usepackage{booktabs}
\usepackage{longtable}
\usepackage{array}
\usepackage{multirow}
\usepackage{tabu}

% -------------------------------------------------
% 7. Backref (referências com página)
% -------------------------------------------------
\usepackage[brazilian,hyperpageref]{backref}

% -------------------------------------------------
% 8. Verbose & blocos de código
% -------------------------------------------------
\usepackage{fvextra}
\DefineVerbatimEnvironment{verbatim}{Verbatim}{
	breaklines=true,
	breakanywhere=true,
	breaksymbolleft={}, 
	breaksymbolright={},
	fontsize=\small
}
% se você quiser que o ambiente Shaded (usado pelo quarto)
% também ganhe fonte menor:
\AtBeginEnvironment{Shaded}{\small}

% e se algum chunk acabar usando listings:
\usepackage{listings}
\lstset{
	breaklines=true,
	breakatwhitespace=true,
	postbreak=\mbox{\textcolor{gray}{$\hookrightarrow$}\space},
	basicstyle=\ttfamily\small,
	columns=fullflexible
}
